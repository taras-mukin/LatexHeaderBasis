\documentclass[12pt,a4paper]{article}

\oddsidemargin=-10.4mm
\textwidth=180mm
\topmargin=-20.4mm
\textheight=267mm
\columnsep=30mm
\columnseprule=0.2mm

\usepackage{mathtext, amssymb, amsmath, lscape, enumitem, dsfont, bbm, bbold}
\usepackage[T2A]{fontenc}
\usepackage[utf8]{inputenc}
\usepackage[english,russian]{babel}

\parindent=0pt
\parskip=8pt
\frenchspacing
\pagestyle{empty}

\renewcommand{\Im}{\mathop{\mathrm{Im}}\nolimits}
\renewcommand{\Re}{\mathop{\mathrm{Re}}\nolimits}
\renewcommand{\tg}{\mathop{\mathrm{tg}}\nolimits}
\newcommand{\ov}{\overline}
\newcommand{\so}{\Rightarrow}
\newcommand{\divis}{\mathrel{\raisebox{-1pt}{\rotatebox{90}{...}}}}

% \renewcommand{\bfdefault}{b}
\newcommand{\sbs}{\large \bfseries}
\newcommand{\rl}{\vspace{16pt} \hrule \vspace{8pt}}
\newcommand{\ve}{\boldsymbol{e}}

\begin{document}


\part*{Основные задачи}

\rl
{\sbs Задача 1}

{\itshape Условие: } Корни  уравнения  $x^n = 1$, как действительные, так и комплексные, называются корнями $n$-й степени из  единицы. Проверить, что корни $n$-й степени образуют группу по умножению. (а) Верно ли, что всякий корень 35-й степени из единицы является кубом некоторого корня 35-й степени из единицы? (б) Тот же вопрос про корни 36-й степени из единицы.

Пусть $X_n =\lbrace\, x\mid x^n = 1 \,\rbrace$. Проверим, является ли $X_n$ группой по умножению. Для этого проверим выполнение аксиом группы:
\\1. \textit{Ассоциативность. }
\\ Если $x_1^n = 1$, $x_2^n = 1$, $x_3^n = 1$, то очевидно $(x_1^n x_2^n) x_3^n=x_1^n (x_2 x_3)^n=(x_1 x_2 x_3)^n=1$.
\\2. \textit{Аксиома единицы. }
\\$\exists$ $e = 1 \in X_n$  $\forall x \in X_n: x*\ve = \ve*x=x$
\\3. \textit{Обратный элемент.}
\\Если $x_1^n=1$, то $\displaystyle \frac{1}{x_1^n}=\left( \frac{1}{x_1}\right)^n=1.$ Значит, для $x_i$ обратным элементом является $\displaystyle \frac{1}{x_i}$.
\\4. \textit{Коммутативность.}
\\ Если $x_1^n = 1$, $x_2^n = 1$, то $(x_1 x_2)^n=(x_2 x_1)^n=1.$
\\ Итак, выполнены аксиомы группы, а также свойство коммуникативности. Значит, \\$X_n$~--- \textbf{абелева группа}.

\textbf{a)} $x^ {35} = 1$
\\ Решения этого уравнения: $x_k=\ve^{i\varphi}$, где $\varphi=\frac{2\pi k}{35}, k={0, 1,\ldots, 34}$.
\\Положим, $x_k=x_m^3$. Тогда $\displaystyle \ve^{\frac{2i\pi k}{35}}=(\ve^{\frac{2i\pi m}{35}})^{3}=\ve^{\frac{6\pi m}{35}}=1 \Rightarrow \ve^{\frac{2i \pi (3m-k)}{35}}=1 \Rightarrow \\3m-k=35p,  p\in \mathds{Z}$
\\ Значит, $(35p + k) \divis 3$.
\\ 1) $p = 3s \Rightarrow k = 3l$.
\\ 2) $p = 3s + 1 \Rightarrow k = 3l +1 $.
\\ 3) $p = 3s + 2 \Rightarrow k = 3l +2 $.
\\ То есть k может принимать любые значения от 0 до 34. А значит, всякий корень уравнения  $x^ {35} = 1$ является кубом для некоторого другого корня этого уравнения.

{\itshape Ответ: } верно.

\textbf{б)} $x^ {36} = 1$
\\Положим, $x_k=x_m^3$. Тогда $\displaystyle \ve^{\frac{2i\pi k}{36}}=(\ve^{\frac{2i\pi m}{36}})^{3}=\ve^{\frac{6\pi m}{36}}=1$ $\Rightarrow$ $\ve^{\frac{2i \pi (3m-k)}{36}}=1$ $\Rightarrow \\3m-k=6p,  p\in \mathds{Z}$
\\ Откуда получим, что $k = 3l$. А значит, не все корни уравнения $x^ {36} = 1$ являются кубами некоторого другого корня этого уравнения.

{\itshape Ответ: } неверно.





\rl
{\sbs Задача 2}

{\itshape Условие: } $C_{360}$~--- цикличекая группа порядка 360. Найти число решений уравнения $x^k=\boldsymbol{e}$ и количество элементов порядка $k$ в группе $C_{360}$ при а)\,$k = 7$; б)\,$k = 12$; в)\,$k = 48$. Сколько в $C_{360}$ порождающих элементов?

а) Порядок элемента~--- делитель порядка группы, а значит элементов порядка 7 нет. $(7, 360) = 1$, значит уравнение $x^7 = \boldsymbol{e}$ имеет единственное решение: $x = \boldsymbol{e}$.

б) Количество элементов порядка 12 равно: $\varphi (12) = 4$. Т.\,к. $x^{12} = (x^2)^6 = (x^3)^4 = (x^4)^3 = (x^6)^2$, то количество решений уравнения $x^{12} = \boldsymbol{e}$ равно количеству элементов порядка 1, 2, 3, 4, 6, 12. Тогда число решений равно $1 + \varphi(2) + \varphi(3) + \varphi(4) + \varphi(6) + \varphi(12) = 12$ .\par
в) Количество элементов порядка 48 равно 0, т.\,к. $360$ не делится на $48$. Аналогично пункту б) количество решений уравнения $x^{48} = \boldsymbol{e}$ равно:\\
$1 + \varphi(2) + \varphi(3) + \varphi(4) + \varphi(6) + \varphi(8) + \varphi(12) + \varphi(24) = 24$.\par
Количество порождающих элементов~--- количество элементов порядка 360:\\
$\varphi(360) = \varphi(72)\cdot\varphi(5) =\varphi(8)\cdot\varphi(9)\cdot\varphi(5) = 96$.

{\itshape Ответ: } а)\,1; б)\,12; в)\,24; порождающих элементов 96.



\end{document}
