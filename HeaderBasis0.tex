\documentclass[12pt,a4paper]{article}

\oddsidemargin=-10.4mm
\textwidth=180mm
\topmargin=-20.4mm
\textheight=267mm
\columnsep=30mm
\columnseprule=0.2mm

\usepackage{mathtext, amssymb, amsmath, lscape, enumitem, dsfont, bbm, bbold}
\usepackage[T2A]{fontenc}
\usepackage[utf8]{inputenc}
\usepackage[english,russian]{babel}

\parindent=0pt
\parskip=8pt
\frenchspacing
\pagestyle{empty}

\renewcommand{\Im}{\mathop{\mathrm{Im}}\nolimits}
\renewcommand{\Re}{\mathop{\mathrm{Re}}\nolimits}
\renewcommand{\tg}{\mathop{\mathrm{tg}}\nolimits}
\newcommand{\ov}{\overline}
\newcommand{\so}{\Rightarrow}
\newcommand{\divis}{\mathrel{\raisebox{-1pt}{\rotatebox{90}{...}}}}

% \renewcommand{\bfdefault}{b}
\newcommand{\sbs}{\large \bfseries}
\newcommand{\rl}{\vspace{16pt} \hrule \vspace{8pt}}
\newcommand{\ve}{\boldsymbol{e}}

\begin{document}


\part*{Основные задачи}

\rl
{\sbs Задача 1}

{\itshape Условие: } Корни  уравнения  $x^n = 1$, как действительные, так и комплексные, называются корнями $n$-й степени из  единицы. Проверить, что корни $n$-й степени образуют группу по умножению. (а) Верно ли, что всякий корень 35-й степени из единицы является кубом некоторого корня 35-й степени из единицы? (б) Тот же вопрос про корни 36-й степени из единицы.

Пусть $X_n =\lbrace\, x\mid x^n = 1 \,\rbrace$. Проверим, является ли $X_n$ группой по умножению. Для этого проверим выполнение аксиом группы:
\\1. \textit{Ассоциативность. }
\\ Если $x_1^n = 1$, $x_2^n = 1$, $x_3^n = 1$, то очевидно $(x_1^n x_2^n) x_3^n=x_1^n (x_2 x_3)^n=(x_1 x_2 x_3)^n=1$.
\\2. \textit{Аксиома единицы. }
\\$\exists$ $e = 1 \in X_n$  $\forall x \in X_n: x*\ve = \ve*x=x$
\\3. \textit{Обратный элемент.}
\\Если $x_1^n=1$, то $\displaystyle \frac{1}{x_1^n}=\left( \frac{1}{x_1}\right)^n=1.$ Значит, для $x_i$ обратным элементом является $\displaystyle \frac{1}{x_i}$.
\\4. \textit{Коммутативность.}
\\ Если $x_1^n = 1$, $x_2^n = 1$, то $(x_1 x_2)^n=(x_2 x_1)^n=1.$
\\ Итак, выполнены аксиомы группы, а также свойство коммуникативности. Значит, \\$X_n$~--- \textbf{абелева группа}.

\textbf{a)} $x^ {35} = 1$
\\ Решения этого уравнения: $x_k=\ve^{i\varphi}$, где $\varphi=\frac{2\pi k}{35}, k={0, 1,\ldots, 34}$.
\\Положим, $x_k=x_m^3$. Тогда $\displaystyle \ve^{\frac{2i\pi k}{35}}=(\ve^{\frac{2i\pi m}{35}})^{3}=\ve^{\frac{6\pi m}{35}}=1 \Rightarrow \ve^{\frac{2i \pi (3m-k)}{35}}=1 \Rightarrow \\3m-k=35p,  p\in \mathds{Z}$
\\ Значит, $(35p + k) \divis 3$.
\\ 1) $p = 3s \Rightarrow k = 3l$.
\\ 2) $p = 3s + 1 \Rightarrow k = 3l +1 $.
\\ 3) $p = 3s + 2 \Rightarrow k = 3l +2 $.
\\ То есть k может принимать любые значения от 0 до 34. А значит, всякий корень уравнения  $x^ {35} = 1$ является кубом для некоторого другого корня этого уравнения.

{\itshape Ответ: } верно.

\textbf{б)} $x^ {36} = 1$
\\Положим, $x_k=x_m^3$. Тогда $\displaystyle \ve^{\frac{2i\pi k}{36}}=(\ve^{\frac{2i\pi m}{36}})^{3}=\ve^{\frac{6\pi m}{36}}=1$ $\Rightarrow$ $\ve^{\frac{2i \pi (3m-k)}{36}}=1$ $\Rightarrow \\3m-k=6p,  p\in \mathds{Z}$
\\ Откуда получим, что $k = 3l$. А значит, не все корни уравнения $x^ {36} = 1$ являются кубами некоторого другого корня этого уравнения.

{\itshape Ответ: } неверно.





\rl
{\sbs Задача 2}

{\itshape Условие: } $C_{360}$~--- цикличекая группа порядка 360. Найти число решений уравнения $x^k=\boldsymbol{e}$ и количество элементов порядка $k$ в группе $C_{360}$ при а)\,$k = 7$; б)\,$k = 12$; в)\,$k = 48$. Сколько в $C_{360}$ порождающих элементов?

а) Порядок элемента~--- делитель порядка группы, а значит элементов порядка 7 нет. $(7, 360) = 1$, значит уравнение $x^7 = \boldsymbol{e}$ имеет единственное решение: $x = \boldsymbol{e}$.

б) Количество элементов порядка 12 равно: $\varphi (12) = 4$. Т.\,к. $x^{12} = (x^2)^6 = (x^3)^4 = (x^4)^3 = (x^6)^2$, то количество решений уравнения $x^{12} = \boldsymbol{e}$ равно количеству элементов порядка 1, 2, 3, 4, 6, 12. Тогда число решений равно $1 + \varphi(2) + \varphi(3) + \varphi(4) + \varphi(6) + \varphi(12) = 12$ .\par
в) Количество элементов порядка 48 равно 0, т.\,к. $360$ не делится на $48$. Аналогично пункту б) количество решений уравнения $x^{48} = \boldsymbol{e}$ равно:\\
$1 + \varphi(2) + \varphi(3) + \varphi(4) + \varphi(6) + \varphi(8) + \varphi(12) + \varphi(24) = 24$.\par
Количество порождающих элементов~--- количество элементов порядка 360:\\
$\varphi(360) = \varphi(72)\cdot\varphi(5) =\varphi(8)\cdot\varphi(9)\cdot\varphi(5) = 96$.

{\itshape Ответ: } а)\,1; б)\,12; в)\,24; порождающих элементов 96.




\rl
{\sbs Задача 3}

{\itshape Условие: } veni vidi vici

Здесь будет написано решение

{\itshape Ответ: } А тут будет ответ




\rl
{\sbs Задача 4}

{\itshape Условие: } veni vidi vici

Здесь будет написано решение

{\itshape Ответ: } А тут будет ответ




\rl
{\sbs Задача 5}

{\itshape Условие: } veni vidi vici

Здесь будет написано решение

{\itshape Ответ: } А тут будет ответ




\rl
{\sbs Задача 6}

{\itshape Условие: } veni vidi vici

Здесь будет написано решение

{\itshape Ответ: } А тут будет ответ




\rl
{\sbs Задача 7}

{\itshape Условие: } Порождают ли перестановки порядка 11 группу $S_{11}$?

Как известно, любая перестановка разбиватся на циклы. Если это циклы длин $a_1, a_2,\ldots, a_k$, то порядком перестановки будет $(a_1,a_2,\ldots,a_k)$. Отсюда следует, что элементы порядка 11 в $S_{11}$ - это циклы длины 11. Каждый такой цикл представляется в виде 10 композиций: цикл $(b_1,b_2,\ldots,b_{11})$ представляется как $(b_1,b_2)\circ (b_2,b_3)\circ\ldots\circ (b_{10},b_{11})$. Таким образом, данные перестановки порождают лишь перестановки, выражающиеся четным количеством транспозиций, и, например, траспозиция $(1,2)$ не может быть получена.

{\itshape Ответ:} не порождают.





\rl
{\sbs Задача 8}

{\itshape Условие: } Построить некоммутативную группу группу минимального порядка.

Рассмотрим различные значения порядков.

$n=1$: Очевидно, что это группа из одной единицы. Такая группа абелева.

$n=2$: Пусть $a$~--- элемент группы. Тогда $a*\boldsymbol{e} = \boldsymbol{e}*a = \boldsymbol{e}$ и получаем, что группа абелева.

$n=3$: Пусть $G = \{\boldsymbol{e}, a ,b\}$. Чтобы доказать что группа абелева достаточно показать что $a * b = b * a$. Посмотрим чему может быть равно $a*b$. Если $a*b=a$, то $b=\boldsymbol{e}$. Аналогично, если $a*b=b$, то $a=\boldsymbol{e}$. Получаем, что $a*b=\boldsymbol{e}$. Аналогично, $b*a=\boldsymbol{e}$ и группа абелева.

$n=4$: Пусть $G = \{\boldsymbol{e}, a, b, c\}$. Посмотрим чему может быть равен порядок элементов. Порядок элемента должен делить порядок группы, то есть он равен или 2 или 4. Если порядок равен 4 (порядку группы), то группа циклическая, а этот элемент~--- порождающий. Циклическая группа является абелевой. Порядок равный единице означает, что элемент равен $\boldsymbol{e}$. Получаем, что порядок любого элемента равен 2. Тогда
\begin{gather*}
(ab)^2=\boldsymbol{e} \Rightarrow (ab)(ab)=\boldsymbol{e} \\
(ab)(ba)=ab^2a=a^2=\boldsymbol{e}=(ab)(ab)\Rightarrow \\
(ab)(ba)=(ab)(ab)\Rightarrow ab=ba
\end{gather*}
Получили, что группа абелева.

$n=5$: Порядок элемента делит порядок группы, то есть порядок элемента в группе порядка 5 может быть равен только 5, что означает, что группа всегда циклическая и абелева.

$n=6$: Пример группы из 6 элементов это $S_3$. (Число элементов равно $3!=6$)

{\itshape Ответ: } $S_3$




\rl
{\sbs Задача 9}

{\itshape Условие: } veni vidi vici

Здесь будет написано решение

{\itshape Ответ: } А тут будет ответ




\rl
{\sbs Задача 10}

{\itshape Условие: } veni vidi vici

Здесь будет написано решение

{\itshape Ответ: } А тут будет ответ




\rl
{\sbs Задача 11}

{\itshape Условие: } Доказать, что группа вращений трехмерного куба изоморфна группе $S_4$.

Каждому вращению куба соответствует перестановка его четырех диагоналей. Композиции вращений соответствует композиция перестановок. Разным вращениям  $a$ и $b$ соответствуют различные перестановки, т.\,к. в ином случае нетождественному вращению  $ab^{-1}$ соответствовала бы тождественная перестановка. Каждой перестановке соответствует только одно вращение.

Таким образом, установлен установлен изоморфизм вращения группе $S_4$.




\rl
{\sbs Задача 12}

{\itshape Условие: } veni vidi vici

Здесь будет написано решение

{\itshape Ответ: } А тут будет ответ




\rl
{\sbs Задача 13}

{\itshape Условие: } veni vidi vici

Здесь будет написано решение

{\itshape Ответ: } А тут будет ответ




\rl
{\sbs Задача 14}

{\itshape Условие: } veni vidi vici

Здесь будет написано решение

{\itshape Ответ: } А тут будет ответ




\rl
{\sbs Задача 15}

{\itshape Условие: } veni vidi vici

Здесь будет написано решение

{\itshape Ответ: } А тут будет ответ




\rl
{\sbs Задача 16}

{\itshape Условие: } veni vidi vici

Здесь будет написано решение

{\itshape Ответ: } А тут будет ответ




\rl
{\sbs Задача 17}

{\itshape Условие: } veni vidi vici

Здесь будет написано решение

{\itshape Ответ: } А тут будет ответ




\rl
{\sbs Задача 18}

{\itshape Условие: } veni vidi vici

Здесь будет написано решение

{\itshape Ответ: } А тут будет ответ




\rl
{\sbs Задача 19}

{\itshape Условие: } Является ли кольцом главных идеалов кольцо $\mathds{Z}_{72}$?

Пусть $I$~--- произвольный идеал кольца $R$. Нужно доказать, что $I$ имеет вид $n\mathds{Z}_{72}$.

Если $I = \{0\}$, то $I  = 0\mathds{Z}_{72}$.
Пусть $I \neq \{0\}$. Тогда пусть $n$~--- наименьший элемент из $I$.

$n \mathds{Z} \subseteq I$ в силу определения идеала.\null

Пусть $a \in I$. Разделим $a$ на $n$. $r = a - qn,  r \in I$. Если $a \notin n \mathds{Z}$, то $r>0$ и $r<n$. Получаем противоречие ($n$~--- наименьший элемент идеала) и $a \in  n\mathds{Z}$. Значит $I = n\mathds{Z}$, то есть все идеалы главные.

{\itshape Ответ: } является.




\rl
{\sbs Задача 20}

{\itshape Условие: } Решить диофантово уравнение: $33x+23y=4$.

$33x+23y=4$
\\$23(x+y)+10x=4$
\\\textit{Замена}: $x+y=z$
\\$23z+10x=4$
\\$10(2z+x)+3z=4$
\\\textit{Замена}: $2z+x=a$
\\$10a+3z=4$
\\$3(3a+z)+a=4$
\\\textit{Замена}: $3a+z=b$
\\$3b+a=4$
\\Заметим, что
$
\left\{
\begin{array}{l}
3b\equiv 0 \pmod{3}\\
4\equiv 1 \pmod {3}
\end{array}
\right.
$
$\Longrightarrow $ $a\equiv 1 \pmod {3}$.
\\Откуда
$
\left\{
\begin{array}{l}
a=3n+1\\
b=1-n
\end{array}
\right.
$
\null
\\Проведя ряд обратных замен, выразим $x$ и $y$ через $n$:
$
\left\{
\begin{array}{l}
x = 23 n + 5\\
y = - 33 n - 7
\end{array}
\right.
$
, где $n\in \mathds{Z}$

{\itshape Ответ: } $x=23n-5, y=-33n-7, n\in \mathds Z$.




\rl
{\sbs Задача 21}

{\itshape Условие: } veni vidi vici

Здесь будет написано решение

{\itshape Ответ: } А тут будет ответ




\rl
{\sbs Задача 22}

{\itshape Условие: } veni vidi vici

Здесь будет написано решение

{\itshape Ответ: } А тут будет ответ




\rl
{\sbs Задача 23}

{\itshape Условие: } veni vidi vici

Здесь будет написано решение

{\itshape Ответ: } А тут будет ответ




\rl
{\sbs Задача 24}

{\itshape Условие: } veni vidi vici

Здесь будет написано решение

{\itshape Ответ: } А тут будет ответ




\rl
{\sbs Задача 25}

{\itshape Условие: } veni vidi vici

Здесь будет написано решение

{\itshape Ответ: } А тут будет ответ




\rl
{\sbs Задача 26}

{\itshape Условие: } veni vidi vici

Здесь будет написано решение

{\itshape Ответ: } А тут будет ответ




\rl
{\sbs Задача 27}

{\itshape Условие: } Многочлен $f(x)$ над полем  $F_5$ степени  $2$ принимает значение $1$ в точке $1$, значение $2$ в точке $3$ и значение $3$ в точке $4$. Найти $f(x)$.

$f(x)=ax^2+bx+c, \quad a, b ,c \in F_5$

$x=1 \colon \quad a+b+c  \equiv 1\pmod{5};$

$x=3 \colon \quad 9a+3b+c  \equiv 2\pmod{5};$

$x=4 \colon \quad 16a+4b+c \equiv 3\pmod{5};$


$\left\{
\begin{aligned}
a+b+c & \equiv 1\pmod{5}; \\
9a+3b+c & \equiv 2\pmod{5}; \\
16a+4b+c & \equiv 3\pmod{5}.
\end{aligned}
\right.$
$\quad \Leftrightarrow \quad $
$\left\{
\begin{aligned}
a+b+c & \equiv 1\pmod{5}; \\
9a+3b+c & \equiv 2\pmod{5}; \\
7a+b & \equiv 1\pmod{5}.
\end{aligned}
\right.
\quad \Leftrightarrow \quad $
$\left\{
\begin{aligned}
a+b+c & \equiv 1\pmod{5}; \\
a+b & \equiv 0\pmod{5}; \\
7a+b & \equiv 1\pmod{5}.
\end{aligned}
\right.
\quad \Leftrightarrow \quad $

$\left\{
\begin{aligned}
c & \equiv 1\pmod{5}; \\
a+b & \equiv 0\pmod{5}; \\
6a & \equiv 1\pmod{5}.
\end{aligned}
\right.$
$\quad \Leftrightarrow \quad $
$\left\{
\begin{aligned}
a &= 1; \\
b &=4; \\
c &=1.
\end{aligned}
\right.$

{\itshape Ответ: } 1, 4, 1.




\rl
{\sbs Задача 28}

{\itshape Условие: } а)\,Найти сумму всех квадратичных вычетов по модулю 73. б)\,Найти произведение всех квадратичных невычетов по модулю 103.

а) По Малой теореме Ферма любой элемент группы является решением уравнения $x^{72} = 1$.\\
$(x^{72} - 1 = 0) \so (x^{36} - 1)(x^{36} + 1) = 0$

Заметим, что все квадратичные вычеты удовлетворяют уравнению $x^{36} - 1 = 0$, т.\,к. если $a$~--- квадратичный вычет, то $a = x^2$, где $x$~--- элемент группы. Тогда все корни, которые содержатся в первой скобке уравнения~--- квадратичные вычеты. Докажем что все кв. вычеты являются корнями уравнения $x^{36} - 1 = 0$. Для этого покажем, что квадратичных вычетов по модулю 73 ровно 36 (не считая 0).

Рассмотрим всевозможные остатки по модулю 73. Очевидно, что элементы $a$ и $(-a)$ образуют один квадратичный вычет. Тогда всего квадратичных вычетов не больше чем $(73 - 1)/2 = 36$. Но их хотя бы 36, а значит их ровно 36. Тогда все корни, которые содержатся в первой скобке~--- квадратичные вычеты, все корни второй скобки~--- квадратичные невычеты. Тогда по теореме Виета сумма квадратичных вычетов равна 0.

б) Из пункта а) следует, что все квадратичные невычеты содержатся во второй скобке, а их произведение по теореме Виета равно (-1).

{\itshape Ответ: } а)\,1; б)\,$-1$.




\rl
{\sbs Задача 29}

{\itshape Условие: } veni vidi vici

Здесь будет написано решение

{\itshape Ответ: } А тут будет ответ




\rl
{\sbs Задача 30}

{\itshape Условие: } Решить уравнение $1+x+x^2+\ldots+x^6\equiv 0\pmod{29}$.

$1+x+x^2+\ldots+x^6\equiv 0\pmod{29}\\
(1+x+x^2+\ldots+x^6)(1-x)\equiv 0\pmod{29}\\
1-x^7\equiv 0\pmod{29}\\
x^7\equiv 1\pmod{29}$

Заметим, что $\forall x: x^{28}\equiv 1\pmod{29}$, так что все вычеты 4 степени будут решениями. Их можно получить простым подсчетом: квадратичными вычетами будут числа 1, 4, 5, 6, 7, 9, 13, 16, 20, 22, 23, 24, 25, 28. Вычетами 4 степени будут тогда все различные квадраты данных чисел~--- 1, 7, 16, 20, 23, 24, 25. Поскольку уравнение 7 степени~--- больше решений нет. Для изначального же уравнения 1 является посторонним корнем (это решение появилось в результате умножения на $1-x$).

{\itshape Ответ:} 7, 16, 20, 23, 24, 25.





\rl
{\sbs Задача 31}

{\itshape Условие: } veni vidi vici

Здесь будет написано решение

{\itshape Ответ: } А тут будет ответ

































\newpage
\part*{Дополнительные задачи}


\rl
{\sbs Дополнительная задача 1}

{\itshape Условие: } veni vidi vici

Здесь будет написано решение

{\itshape Ответ: } А тут будет ответ





\rl
{\sbs Дополнительная задача 2}

{\itshape Условие: } $C_{360}$~--- цикличекая группа порядка 360. Найти число решений уравнения $x^k=\boldsymbol{e}$ и количество элементов порядка $k$ в группе $C_{360}$ при а)\,$k = 7$; б)\,$k = 12$; в)\,$k = 48$. Сколько в $C_{360}$ порождающих элементов?


{\itshape Ответ: } а)\,1; б)\,12; в)\,24; порождающих элементов 96.




\rl
{\sbs Дополнительная задача 3}

{\itshape Условие: } veni vidi vici

Здесь будет написано решение

{\itshape Ответ: } А тут будет ответ




\rl
{\sbs Дополнительная задача 4}

{\itshape Условие: } veni vidi vici

Здесь будет написано решение

{\itshape Ответ: } А тут будет ответ




\rl
{\sbs Дополнительная задача 5}

{\itshape Условие: } veni vidi vici

Здесь будет написано решение

{\itshape Ответ: } А тут будет ответ




\rl
{\sbs Дополнительная задача 6}

{\itshape Условие: } veni vidi vici

Здесь будет написано решение

{\itshape Ответ: } А тут будет ответ




\rl
{\sbs Дополнительная задача 7}

{\itshape Условие: } Порождают ли перестановки порядка 11 группу $S_{11}$?



{\itshape Ответ:} не порождают.





\rl
{\sbs Дополнительная задача 8}

{\itshape Условие: } Построить некоммутативную группу группу минимального порядка.



{\itshape Ответ: } $S_3$




\rl
{\sbs Дополнительная задача 9}

{\itshape Условие: } veni vidi vici

Здесь будет написано решение

{\itshape Ответ: } А тут будет ответ




\rl
{\sbs Дополнительная задача 10}

{\itshape Условие: } Построить некоммутативную группу порядка 8, все подгрупы которой нормальны.

Всего 2 некоммутативных групп порядка 8. Рассмотрим группу симметрий четырехугольника $D_4$ и покажем что она удовлетворяет условию.\null

Таблица Кэли:

\begin{center}
\begin{tabular}{|l|l|l|l||l|l|l|l|}
\hline
$e$ & $a$ & $a^2$ & $a^3$ & $b$ & $ab$ & $a^2b$ & $a^3b$\\
\hline
$a$ & $a^2$ & $a^3$ & $e$ & $ab$ & $a^2b$ & $a^3b$ & $b$\\
\hline
$a^2$ & $a^3$ & $e$ & $a$ & $a^2b$ & $a^3b$ & $b$ & $ab$\\
\hline
$a^3$ & $e$ & $a$ & $a^2$ & $a^3b$ & $b$ & $ab$ & $a^2b$\\
\hline
\hline
$b$ & $a^3b$ & $a^2b$ & $ab$ & $e$ & $a^3$ & $a^2$ & $a^2$\\
\hline
$ab$ & $b$ & $a^3b$ & $a^2b$ & $a^2$ & $e$ & $a^3$ & $a^2$\\
\hline
$a^2b$ & $ab$ & $b$ & $a^3b$ & $a^2$ & $a^2$ & $e$ & $a^3$\\
\hline
$a^3b$ & $a^2b$ & $ab$ & $b$ & $a^3$ & $a^2$ & $a^2$ & $e$\\
\hline
\end{tabular}
\end{center}

Очевидно, что группа не является абелевой. Рассмотрим возможные подгруппы этой группы.\null

Пусть порядок подгруппы равен 4. Тогда нужно выбрать 3 элемента (кроме единичного). Ни один из них не может содержать $b$, так как в этом случае при умножении на степень элемента без $b$ получим все элементы, содержащие $b$. Тогда все элементы, содержащие $b$, должны быть в подгруппе, но элементов с $b$ 4, а не единичных элементов подгруппы 3, поэтому существует только одна подгруппа порядка 4. $H_4 = \{\boldsymbol{e}, a, a^2, a^3\}$.\null

Теперь нужно построить смежные классы. Построить левый смежный класс означает, что мы должны взять элементы на пересечении первых четырех столбцов и одной строки. Для правого нужно взять 4 первых строки и 1 столбец. В силу симметрии таблицы строки и столбцы равнозначны, то есть левые и правые смежные классы совпадают, то есть $H_4$~--- нормализатор.\null

Пусть теперь порядок подгруппы равен 2. Таких подгрупп 7. Проводя аналогичные рассуждения, получаем что подгруппы порядка 2 тоже нормализаторы.




\rl
{\sbs Дополнительная задача 11}

{\itshape Условие: } Доказать, что группа вращений трехмерного куба изоморфна группе $S_4$.

{\itshape Ответ: } А тут будет ответ




\rl
{\sbs Дополнительная задача 12}

{\itshape Условие: } veni vidi vici

Здесь будет написано решение

{\itshape Ответ: } А тут будет ответ




\rl
{\sbs Дополнительная задача 13}

{\itshape Условие: } veni vidi vici

Здесь будет написано решение

{\itshape Ответ: } А тут будет ответ




\rl
{\sbs Дополнительная задача 14}

{\itshape Условие: } Доказать, что если порядок абелевой группы $G$ равен $nm$, где $(n,m) = 1$, то $G$ изоморфна прямому произведению групп порядков $n$ и $m$.

Требуется доказать, что в условиях теоремы G изоморфна прямому произведению некоторых групп. В качестве этих групп возьмем подгруппы  $G'$ и $G''$ группы $G_{mn}$.
\textit{Прямым произведением групп} $G'$ и $G''$ называется множество, состоящее из всевозможных элементов вида ($g'_1$,$g''_1$). Заметим, что если в произведении $G_n' \times G_m'' $ элементов меньше, чем $mn$, то построить изоморфизм $G_{mn}$ и $G_n' \times G_m'' $ невозможно.

Покажем, что в нашем случае $|G_{mn}|$ = $|G_n' \times G_m''| $.

Пусть $G_n' = \lbrace e, a_1 ... a_{n-1} \rbrace$, $G_m'' = \lbrace e, b_1 .. b_{m-1}\rbrace.$ Запишем эти элементы в таблицу. На $(i,j)$ месте этой таблицы стоит элемент ($g'_i$,$g''_j$) группы $G_n' \times G_m''$.


\begin{center}
\begin{tabular}{|l|l|l|l|l|l|l|l|}
\hline
$ $ & $\textit{\textbf{e}}$ & $\textit{\textbf{a}}$ & $\textit{\textbf{a}}_2$ & $\textit{\textbf{..}}$ & $\textit{\textbf{a}}_i$ & $\textit{\textbf{..}}$ & $\textit{\textbf{a}}_{n-1}$\\
\hline
$\textit{\textbf{e}}$ & $(e,e)$ & $(a,e)$ & $(a_2,e)$ & $..$ & $(a_i,e)$ & $..$ & $(a_{n-1},e)$\\
\hline
$\textit{\textbf{b}}$ & $(e,b)$ & $(a,b)$ & $(a_2,b)$ & $..$ & $(a_i,b)$ & $..$ & $(a_{n-1},b)$\\
\hline
$\textit{\textbf{b}}_2$ & $(e,b_2)$ & $(a,b_2)$ & $(a_2,b_2)$ & $..$ & $(a_i,b_2)$ & $..$ & $(a_{n-1},b_2)$\\
\hline
$\textit{\textbf{..}}$ & $..$ & $..$ & $..$ & $..$ & $..$ & $..$ & $..$\\
\hline
$\textit{\textbf{b}}_j$ & $(e,b_j)$ & $(a,b_j)$ & $(a_2,b_j)$ & $..$ & $(a_i,b_j)$ & $..$ & $(a_{n-1},b_j)$\\
\hline
$\textit{\textbf{..}}$ & $..$ & $..$ & $..$ & $..$ & $..$ & $..$ & $..$\\
\hline
$\textit{\textbf{b}}_{m-1}$ & $(e,b_{m-1})$ & $(a,b_{m-1})$ & $(a^2,b_{m-1})$ & $..$ & $(a_i,b_{m-1})$ & $..$ & $(a_{n-1},b_{m-1})$\\
\hline
\end{tabular}
\end{center}


Предположим, что элемент $a_i\neq \ve$ группы $G_n' $ равен элементу $b_j \neq \ve$ группы $G_m''$. Тогда, как видно из таблицы, в группе $G_n' \times G_m''$ будут повторяющиеся элементы, а значит, $|G_n' \times G_m''|$ $< mn$. Однако, так как $a_i \in G_n^ \shortmid$, порядок этого элемента делит $n$, с другой стороны $a_i=b_j \in G_m'$, а значит порядок этого элемента делит $m$. По условию $(n,m)=1$, то есть порядок нашего элемента может быть равен только 1, что неверно. Получили противоречие $\Rightarrow$ в  $G_m''$ и $G_n'$ нет равных элементов, кроме $\ve$ $\Rightarrow$ $|G_n' \times G_m''|$ = $|G_{mn}| = mn.$

Построим изоморфизм следующим образом: $(a_i,b_j) \in G_n' \times G_m'' \longrightarrow a_i \ast b_j,$ где $a_i, b_i \in G_{mn}$. При этом среди $a_i\ast b_j$ есть все элементы группы, отображение биективно и групповая операция сохраняется по свойствам прямого произведения: $(a_i,b_j)\ast (a_{i2},b_{j2})=(a_i \ast a_{i2}, b_j \ast b_{j2}).$


Таким образом мы доказали, что если порядок абелевой группы $G$ равен $nm$, где $(n,m) = 1$, то $G$ изоморфна прямому произведению групп порядков $n$ и $m$, ч.\,т.\,д.\,$\blacksquare$




\rl
{\sbs Дополнительная задача 15}

{\itshape Условие: } Группа называется $p$-группой, если ее порядок является степенью простого числа~$p$. Центром группы называется множество элементов, коммутирующих со всеми элементами группы. Доказать, что центр  $p$-группы состоит не только из единичного элемента.

Пусть конечная группа $G$~--- нетривиальная $p$-группа, т.\,е.  $|G|=p^n$ и ее центр $Z$ состоит не только из единичного элемента. Множество $ G\setminus Z$ разбивается на нетривиальные классы сопряженных элементов, число элементов в каждом из которых, согласно формуле  $|C(x)|=\frac{|G|}{|Z(x)|}$, делится на  $p$. Следовательно, число элементов центра также делится на $p$. Таким образом, центр $p$-группы состоит не только из единичного элемента.




\rl
{\sbs Дополнительная задача 16}

{\itshape Условие: } Доказать, что всякая группа порядка $p^2$ абелева.

Пусть $G$~--- группа порядка $p^2$, а $Z$~--- ее центр. Предположим тогда, что $Z\neq G$, что и будет означать существование неабелевой группы порядка $p^2$.

Покажем, что число элементов центра делится на $p$. Разобьем $G$ на классы сопряженных элементов. Каждый из элементов центра образовывает класс из одного элемента, поскольку $a^{-1}z_ia = a^{-1}az_i = z_i$.

Каждый из элементов вне центра образует класс с количеством элементов, больше 1. Рассмотрим сопряжение как действие $G$ на себя. Тогда для каждого $g\in G$ стабилизатором будет служить множество сопряжений $s^{-1}gs$ таких, что $gs = sg$. По лемме Бернсайда получаем, что $\displaystyle |C(x)|=\frac{|G|}{|Z(x)|}$, где $C(x)$~--- орбита $x$ относительно сопряжений, $Z(x)$~--- стабилизатор по сопряжениям. Заметим, что количество элементов в классе совпадает с количеством элементов в орбите любого элемента по сопряжениям. Итак, для любого класса сопряженных элементов $\displaystyle |S_i|=|C(x)|=\frac{|G|}{|Z(x)|}\divis p$.

Итак, пусть в центре элементов $t$, тогда, поскольку вся группа разбивается на непересекающиеся классы сопряженных элементов, $p^n = t + pk_1 + p^k_2 +\ldots+ p^k_m$, откуда следует, что $t\divis p$, что и требовалось.

{\footnotesize \slshape До этого момента было написано понадобившееся более полное и громоздкое решение Д15.}

Итак, теперь мы знаем, что $|Z|\divis p$, и значит, по предположению, $|Z|=p$. Тогда $|G/Z|=p$, так что группа $G/Z$ является циклической. Пусть тогда $aZ$~--- ее порождающий элемент. Тогда любой элемент из $G$ представляется в виде $g = a^kz_i$. Но любые два элемента такого вида коммутируют, что противоречит предположению.

Итак, центр такой группы совпадает со всей группой, что и требовалось.




\rl
{\sbs Дополнительная задача 17}

{\itshape Условие: } veni vidi vici

Здесь будет написано решение

{\itshape Ответ: } А тут будет ответ




\rl
{\sbs Дополнительная задача 18}

{\itshape Условие: } veni vidi vici

Здесь будет написано решение

{\itshape Ответ: } А тут будет ответ




\rl
{\sbs Дополнительная задача 19}

{\itshape Условие: } Является кольцом главных идеалов кольцо $\mathds{Z}_{72}$?


{\itshape Ответ: } Является.




\rl
{\sbs Дополнительная задача 20}

{\itshape Условие: } veni vidi vici

Здесь будет написано решение

{\itshape Ответ: } А тут будет ответ




\rl
{\sbs Дополнительная задача 21}

{\itshape Условие: } veni vidi vici

Здесь будет написано решение

{\itshape Ответ: } А тут будет ответ




\rl
{\sbs Дополнительная задача 22}

{\itshape Условие: } veni vidi vici

Здесь будет написано решение

{\itshape Ответ: } А тут будет ответ




\rl
{\sbs Дополнительная задача 23}

{\itshape Условие: } veni vidi vici

Здесь будет написано решение

{\itshape Ответ: } А тут будет ответ




\rl
{\sbs Дополнительная задача 24}

{\itshape Условие: } veni vidi vici

Здесь будет написано решение

{\itshape Ответ: } А тут будет ответ




\rl
{\sbs Дополнительная задача 25}

{\itshape Условие: } veni vidi vici

Здесь будет написано решение

{\itshape Ответ: } А тут будет ответ




\rl
{\sbs Дополнительная задача 26}

{\itshape Условие: } veni vidi vici

Здесь будет написано решение

{\itshape Ответ: } А тут будет ответ




\rl
{\sbs Дополнительная задача 27}

{\itshape Условие: } veni vidi vici

Здесь будет написано решение

{\itshape Ответ: } А тут будет ответ




\rl
{\sbs Дополнительная задача 28}

{\itshape Условие: } а)\,Найти сумму всех квадратичных вычетов по модулю 73. б)\,Найти произведение всех квадратичных невычетов по модулю 103.



{\itshape Ответ: } а)\,1; б)\,$-1$.




\rl
{\sbs Дополнительная задача 29}

{\itshape Условие: } veni vidi vici

Здесь будет написано решение

{\itshape Ответ: } А тут будет ответ




\rl
{\sbs Дополнительная задача 30}

{\itshape Условие: } Решить уравнение $1+x+x^2+\ldots+x^6\equiv 0\pmod{29}$.

{\itshape Ответ:} 7, 16, 20, 23, 24, 25.





\rl
{\sbs Дополнительная задача 31}

{\itshape Условие: } Найти наибольший порядок элемента мультипликативной группы кольца $\mathds Z_{72}$

Мультипликативная группа кольца вычетов по модулю $72$~--- это множество чисел, взаимно простых с $72$, с введенной операцией умножения (той же, что и в кольце вычетов по модулю $72$). Заметим, что число $67$ взаимно просто с $72$, поэтому оно в мультикативной группе. Заметим так же, что по модулю $72$ верны равенства

$ 67*67=25;$

$25*25=49;$

$49*49=25;$

$25*25=49;$

$49*49=25;$

$\qquad \dots$

Ясно, что на $25$ и $49$ мы зациклились, поэтому порядок элемента $67$ равен бесконечности.

{\itshape Ответ: } бесконечность.



\rl
{\sbs Дополнительная задача 32}

{\itshape Условие: } Найти количество нильпотентных элементов в кольце $F_7[x]/(x^{14}+x^7+2)$

$0 = x^{14}+x^7+2 = x^{14}+8x^7 + 16 = (x^7+4)^2$. Заметим теперь, что по МТФ $x^7 \equiv x \pmod{7}$, так что 3 является корнем $x^7+4$, т.е. $(x^7+4)\divis(x-3)=(x+4)$. Проверим, действительно ли $(x+4)^7 = x^7+4$. Действительно, $\displaystyle (x+4)^7 = x^7 + {7\choose 1} 4x^6 + {7\choose 2} 4^2x^5 +\ldots+{7\choose 6} 4^6x + 4^7$. Все коэффициенты, кроме первого и последнего, делятся на 7, и по МТФ $4^7\equiv 4 \pmod 7$, так что $(x+4)^7 = x^7+4^7 = x^7+4$.

Итак, $0 = x^{14}+x^7+2 = (x^7+4)^2=(x+4)^{14}$. Таким образом, все элементы, делящиеся на $x+4$, являются нильпотентами. Все эти элементы вида $(x+4)P(x)$, где $P(x)$ - многочлен из данного кольца степени не выше 12.

Покажем теперь, что среди многочленов, неделящихся на $x+4$, нет нильпотентов. Предоположим, что такой многочлен $R(x)=(x+4)Q(x)+b$, где $b\neq 0$~--- число из $F_7$. Возведем его в 14 степень, по предположению должно получаться 0:

$\displaystyle \bigl((x+4)Q(x)+b\bigr)^{14} = \bigl(Q(x)\bigr)^{14}(x+4)^{14} + {14\choose 1}\bigl(Q(x)\bigr)^{13}(x+4)^{13}b + {14\choose 2}\bigl(Q(x)\bigr)^{12}(x+4)^{12}b^2+\ldots+{14\choose 13}(x+4)Q(x)b^{13} + b^{14} = 0\cdot\bigl(Q(x)\bigr)^{14} + 0\cdot\bigl(Q(x)\bigr)^{13}(x+4)^{13}b + 0\cdot\bigl(Q(x)\bigr)^{12}(x+4)^{12}b^2+\ldots+0\cdot(x+4)Q(x)b^{13} + b^{14} = b^{14}$.

Итак, получили, что $b^{14} = 0$, что равносильно тому, что $b = 0$, что противоречит предположению.

Итак, все нильпотенты вида $(x+4)P(x)$, и их количество есть $7^{13}$.

{\itshape Ответ:} $7^{13}$.




\rl
{\sbs Дополнительная задача 33}

{\itshape Условие: } veni vidi vici

Здесь будет написано решение

{\itshape Ответ: } А тут будет ответ




\rl
{\sbs Дополнительная задача 34}

{\itshape Условие: } veni vidi vici

Здесь будет написано решение

{\itshape Ответ: } А тут будет ответ




\rl
{\sbs Дополнительная задача 35}

{\itshape Условие: } veni vidi vici

Здесь будет написано решение

{\itshape Ответ: } А тут будет ответ




\rl
{\sbs Дополнительная задача 36}

{\itshape Условие: } veni vidi vici

Здесь будет написано решение

{\itshape Ответ: } А тут будет ответ




\rl
{\sbs Дополнительная задача 37}

{\itshape Условие: } veni vidi vici

Здесь будет написано решение

{\itshape Ответ: } А тут будет ответ




\rl
{\sbs Дополнительная задача 38}

{\itshape Условие: } Указать степени неприводимых делителей многочленов а)\,$x^5-2$ из кольца $F_{67}[x]$; б)\,$x^{28}-1$ из кольца $F_3[x]$.

\textbf{Обозначения. }$f_5(x)$~--- исходный многочлен. $f_i(x)$~--- неприводимый многочлен степени $i$.\null

a) Заметим, что только число 41 является корнем многочлена (перебор). Значит $x^5-2=(x-41)f_4(x)$, где $f_4(x)$~--- многочлен 4 степени. Он
может раскладываться как произведение 2 неприводимых многочленов степени 2, или сам $f_4$ является неприводимым. Разделим исходный многочлен на $x-41$. Получим $x^4-26x^3+6x^2-22x+65$. С помощью метода неопределенных коэффициентов находим разложение $(x^2-14x-6)(x^2-12x-22)$. Значит степени неприводимых многочленов равны 1, 2, 2.

б) $x^{28}-1 = (x-1)(x+1)(x^2+1)(...)$. Посмотрим, какие степени могут быть у неприводимых многочленов. Нужно проверить, делится ли $x^{(3^n-1, 28)}-1$ на наш многочлен. Для $n \neq 6, 12, 18, 24$ получаем, что $(3^n-1, 28)$ меньше чем $n$. Так как $x^{28}-1 = (x^7-1)(x^7+1)(x^{14}+1)$, то можно исключить случаи со степенью неприводимых многочленов 24, 18 и 6, 12 и 12. Значит остается два случая для степеней неприводимых многочленов: 12 и 6 и 6, 6 и 6 и 6 и 6.
Многочлен 12 степени это $(x^{14}+1)/(x^2+1) = x^{12}-x^{10}+x^8-x^6+x^4-x^2+1$. Заметим, что неприводимый многочлен степени 6 это  $(x^7+1)/(x+1) = x^6-x^5+x^4-x^3+x^2-x^1+1$. Так как этот многочлен неприводим, что многочлен $(x^6)^2-(x^5)^2+(x^4)^2-(x^3)^2+(x^2)^2-(x^1)^2+1 = x^{12}-x^{10}+x^8-x^6+x^4-x^2+1$. Значит степени неприводимых многочленов равны 1, 1, 2, 6, 6, 12.

{\itshape Ответ: } а)\,1, 2, 2; б)\,1, 1, 2, 6, 6, 12.




\rl
{\sbs Дополнительная задача 39}

{\itshape Условие: } veni vidi vici

Здесь будет написано решение

{\itshape Ответ: } А тут будет ответ





\end{document}
